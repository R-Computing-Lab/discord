\documentclass[10pt,a4paper,onecolumn]{article}
\usepackage{marginnote}
\usepackage{graphicx}
\usepackage{xcolor}
\usepackage{authblk,etoolbox}
\usepackage{titlesec}
\usepackage{calc}
\usepackage{tikz}
\usepackage{hyperref}
\hypersetup{colorlinks,breaklinks,
            urlcolor=[rgb]{0.0, 0.5, 1.0},
            linkcolor=[rgb]{0.0, 0.5, 1.0}}
\usepackage{caption}
\usepackage{tcolorbox}
\usepackage{amssymb,amsmath}
\usepackage{ifxetex,ifluatex}
\usepackage{seqsplit}
\usepackage{fixltx2e} % provides \textsubscript
\usepackage[
  backend=biber,
%  style=alphabetic,
%  citestyle=numeric
]{biblatex}
\bibliography{discord-paper.bib}


% --- Page layout -------------------------------------------------------------
\usepackage[top=3.5cm, bottom=3cm, right=1.5cm, left=1.0cm,
            headheight=2.2cm, reversemp, includemp, marginparwidth=4.5cm]{geometry}

% --- Default font ------------------------------------------------------------
% \renewcommand\familydefault{\sfdefault}

% --- Style -------------------------------------------------------------------
\renewcommand{\bibfont}{\small \sffamily}
\renewcommand{\captionfont}{\small\sffamily}
\renewcommand{\captionlabelfont}{\bfseries}

% --- Section/SubSection/SubSubSection ----------------------------------------
\titleformat{\section}
  {\normalfont\sffamily\Large\bfseries}
  {}{0pt}{}
\titleformat{\subsection}
  {\normalfont\sffamily\large\bfseries}
  {}{0pt}{}
\titleformat{\subsubsection}
  {\normalfont\sffamily\bfseries}
  {}{0pt}{}
\titleformat*{\paragraph}
  {\sffamily\normalsize}


% --- Header / Footer ---------------------------------------------------------
\usepackage{fancyhdr}
\pagestyle{fancy}
\fancyhf{}
%\renewcommand{\headrulewidth}{0.50pt}
\renewcommand{\headrulewidth}{0pt}
\fancyhead[L]{\hspace{-0.75cm}\includegraphics[width=5.5cm]{C:/Users/smaso/Documents/R/win-library/4.0/rticles/rmarkdown/templates/joss/resources/JOSS-logo.png}}
\fancyhead[C]{}
\fancyhead[R]{}
\renewcommand{\footrulewidth}{0.25pt}

\fancyfoot[L]{\footnotesize{\sffamily Trattner and
Garrison, (2021). \{discord\}: An R Package for Discordant-Kinship
Regressions. \textit{Journal of Open Source Software}, (), . \href{https://doi.org/}{https://doi.org/}}}


\fancyfoot[R]{\sffamily \thepage}
\makeatletter
\let\ps@plain\ps@fancy
\fancyheadoffset[L]{4.5cm}
\fancyfootoffset[L]{4.5cm}

% --- Macros ---------

\definecolor{linky}{rgb}{0.0, 0.5, 1.0}

\newtcolorbox{repobox}
   {colback=red, colframe=red!75!black,
     boxrule=0.5pt, arc=2pt, left=6pt, right=6pt, top=3pt, bottom=3pt}

\newcommand{\ExternalLink}{%
   \tikz[x=1.2ex, y=1.2ex, baseline=-0.05ex]{%
       \begin{scope}[x=1ex, y=1ex]
           \clip (-0.1,-0.1)
               --++ (-0, 1.2)
               --++ (0.6, 0)
               --++ (0, -0.6)
               --++ (0.6, 0)
               --++ (0, -1);
           \path[draw,
               line width = 0.5,
               rounded corners=0.5]
               (0,0) rectangle (1,1);
       \end{scope}
       \path[draw, line width = 0.5] (0.5, 0.5)
           -- (1, 1);
       \path[draw, line width = 0.5] (0.6, 1)
           -- (1, 1) -- (1, 0.6);
       }
   }

% --- Title / Authors ---------------------------------------------------------
% patch \maketitle so that it doesn't center
\patchcmd{\@maketitle}{center}{flushleft}{}{}
\patchcmd{\@maketitle}{center}{flushleft}{}{}
% patch \maketitle so that the font size for the title is normal
\patchcmd{\@maketitle}{\LARGE}{\LARGE\sffamily}{}{}
% patch the patch by authblk so that the author block is flush left
\def\maketitle{{%
  \renewenvironment{tabular}[2][]
    {\begin{flushleft}}
    {\end{flushleft}}
  \AB@maketitle}}
\makeatletter
\renewcommand\AB@affilsepx{ \protect\Affilfont}
%\renewcommand\AB@affilnote[1]{{\bfseries #1}\hspace{2pt}}
\renewcommand\AB@affilnote[1]{{\bfseries #1}\hspace{3pt}}
\makeatother
\renewcommand\Authfont{\sffamily\bfseries}
\renewcommand\Affilfont{\sffamily\small\mdseries}
\setlength{\affilsep}{1em}


\ifnum 0\ifxetex 1\fi\ifluatex 1\fi=0 % if pdftex
  \usepackage[T1]{fontenc}
  \usepackage[utf8]{inputenc}

\else % if luatex or xelatex
  \ifxetex
    \usepackage{mathspec}
  \else
    \usepackage{fontspec}
  \fi
  \defaultfontfeatures{Ligatures=TeX,Scale=MatchLowercase}

\fi
% use upquote if available, for straight quotes in verbatim environments
\IfFileExists{upquote.sty}{\usepackage{upquote}}{}
% use microtype if available
\IfFileExists{microtype.sty}{%
\usepackage{microtype}
\UseMicrotypeSet[protrusion]{basicmath} % disable protrusion for tt fonts
}{}

\usepackage{hyperref}
\hypersetup{unicode=true,
            pdftitle={\{discord\}: An R Package for Discordant-Kinship Regressions},
            pdfborder={0 0 0},
            breaklinks=true}
\urlstyle{same}  % don't use monospace font for urls
\usepackage{color}
\usepackage{fancyvrb}
\newcommand{\VerbBar}{|}
\newcommand{\VERB}{\Verb[commandchars=\\\{\}]}
\DefineVerbatimEnvironment{Highlighting}{Verbatim}{commandchars=\\\{\}}
% Add ',fontsize=\small' for more characters per line
\usepackage{framed}
\definecolor{shadecolor}{RGB}{248,248,248}
\newenvironment{Shaded}{\begin{snugshade}}{\end{snugshade}}
\newcommand{\AlertTok}[1]{\textcolor[rgb]{0.94,0.16,0.16}{#1}}
\newcommand{\AnnotationTok}[1]{\textcolor[rgb]{0.56,0.35,0.01}{\textbf{\textit{#1}}}}
\newcommand{\AttributeTok}[1]{\textcolor[rgb]{0.77,0.63,0.00}{#1}}
\newcommand{\BaseNTok}[1]{\textcolor[rgb]{0.00,0.00,0.81}{#1}}
\newcommand{\BuiltInTok}[1]{#1}
\newcommand{\CharTok}[1]{\textcolor[rgb]{0.31,0.60,0.02}{#1}}
\newcommand{\CommentTok}[1]{\textcolor[rgb]{0.56,0.35,0.01}{\textit{#1}}}
\newcommand{\CommentVarTok}[1]{\textcolor[rgb]{0.56,0.35,0.01}{\textbf{\textit{#1}}}}
\newcommand{\ConstantTok}[1]{\textcolor[rgb]{0.00,0.00,0.00}{#1}}
\newcommand{\ControlFlowTok}[1]{\textcolor[rgb]{0.13,0.29,0.53}{\textbf{#1}}}
\newcommand{\DataTypeTok}[1]{\textcolor[rgb]{0.13,0.29,0.53}{#1}}
\newcommand{\DecValTok}[1]{\textcolor[rgb]{0.00,0.00,0.81}{#1}}
\newcommand{\DocumentationTok}[1]{\textcolor[rgb]{0.56,0.35,0.01}{\textbf{\textit{#1}}}}
\newcommand{\ErrorTok}[1]{\textcolor[rgb]{0.64,0.00,0.00}{\textbf{#1}}}
\newcommand{\ExtensionTok}[1]{#1}
\newcommand{\FloatTok}[1]{\textcolor[rgb]{0.00,0.00,0.81}{#1}}
\newcommand{\FunctionTok}[1]{\textcolor[rgb]{0.00,0.00,0.00}{#1}}
\newcommand{\ImportTok}[1]{#1}
\newcommand{\InformationTok}[1]{\textcolor[rgb]{0.56,0.35,0.01}{\textbf{\textit{#1}}}}
\newcommand{\KeywordTok}[1]{\textcolor[rgb]{0.13,0.29,0.53}{\textbf{#1}}}
\newcommand{\NormalTok}[1]{#1}
\newcommand{\OperatorTok}[1]{\textcolor[rgb]{0.81,0.36,0.00}{\textbf{#1}}}
\newcommand{\OtherTok}[1]{\textcolor[rgb]{0.56,0.35,0.01}{#1}}
\newcommand{\PreprocessorTok}[1]{\textcolor[rgb]{0.56,0.35,0.01}{\textit{#1}}}
\newcommand{\RegionMarkerTok}[1]{#1}
\newcommand{\SpecialCharTok}[1]{\textcolor[rgb]{0.00,0.00,0.00}{#1}}
\newcommand{\SpecialStringTok}[1]{\textcolor[rgb]{0.31,0.60,0.02}{#1}}
\newcommand{\StringTok}[1]{\textcolor[rgb]{0.31,0.60,0.02}{#1}}
\newcommand{\VariableTok}[1]{\textcolor[rgb]{0.00,0.00,0.00}{#1}}
\newcommand{\VerbatimStringTok}[1]{\textcolor[rgb]{0.31,0.60,0.02}{#1}}
\newcommand{\WarningTok}[1]{\textcolor[rgb]{0.56,0.35,0.01}{\textbf{\textit{#1}}}}
\usepackage{graphicx,grffile}
\makeatletter
\def\maxwidth{\ifdim\Gin@nat@width>\linewidth\linewidth\else\Gin@nat@width\fi}
\def\maxheight{\ifdim\Gin@nat@height>\textheight\textheight\else\Gin@nat@height\fi}
\makeatother
% Scale images if necessary, so that they will not overflow the page
% margins by default, and it is still possible to overwrite the defaults
% using explicit options in \includegraphics[width, height, ...]{}
\setkeys{Gin}{width=\maxwidth,height=\maxheight,keepaspectratio}
\IfFileExists{parskip.sty}{%
\usepackage{parskip}
}{% else
\setlength{\parindent}{0pt}
\setlength{\parskip}{6pt plus 2pt minus 1pt}
}
\setlength{\emergencystretch}{3em}  % prevent overfull lines
\providecommand{\tightlist}{%
  \setlength{\itemsep}{0pt}\setlength{\parskip}{0pt}}
\setcounter{secnumdepth}{0}
% Redefines (sub)paragraphs to behave more like sections
\ifx\paragraph\undefined\else
\let\oldparagraph\paragraph
\renewcommand{\paragraph}[1]{\oldparagraph{#1}\mbox{}}
\fi
\ifx\subparagraph\undefined\else
\let\oldsubparagraph\subparagraph
\renewcommand{\subparagraph}[1]{\oldsubparagraph{#1}\mbox{}}
\fi

% Pandoc citation processing
\newlength{\csllabelwidth}
\setlength{\csllabelwidth}{3em}
\newlength{\cslhangindent}
\setlength{\cslhangindent}{1.5em}
% for Pandoc 2.8 to 2.10.1
\newenvironment{cslreferences}%
  {}%
  {\par}
% For Pandoc 2.11+
\newenvironment{CSLReferences}[2] % #1 hanging-ident, #2 entry spacing
 {% don't indent paragraphs
  \setlength{\parindent}{0pt}
  % turn on hanging indent if param 1 is 1
  \ifodd #1 \everypar{\setlength{\hangindent}{\cslhangindent}}\ignorespaces\fi
  % set entry spacing
  \ifnum #2 > 0
  \setlength{\parskip}{#2\baselineskip}
  \fi
 }%
 {}
\usepackage{calc} % for calculating minipage widths
\newcommand{\CSLBlock}[1]{#1\hfill\break}
\newcommand{\CSLLeftMargin}[1]{\parbox[t]{\csllabelwidth}{#1}}
\newcommand{\CSLRightInline}[1]{\parbox[t]{\linewidth - \csllabelwidth}{#1}\break}
\newcommand{\CSLIndent}[1]{\hspace{\cslhangindent}#1}

\usepackage{booktabs}
\usepackage{longtable}
\usepackage{array}
\usepackage{multirow}
\usepackage{wrapfig}
\usepackage{float}
\usepackage{colortbl}
\usepackage{pdflscape}
\usepackage{tabu}
\usepackage{threeparttable}
\usepackage{threeparttablex}
\usepackage[normalem]{ulem}
\usepackage{makecell}
\usepackage{xcolor}

\title{\{discord\}: An R Package for Discordant-Kinship Regressions}

        \author[1]{Jonathan D. Trattner}
          \author[2]{S. Mason Garrison}
    
      \affil[1]{Department of Neuroscience, Wake Forest School of
Medicine}
      \affil[2]{Department of Psychology, Wake Forest University}
  \date{\vspace{-5ex}}

\begin{document}
\maketitle

\marginpar{
  %\hrule
  \sffamily\small

  {\bfseries DOI:} \href{https://doi.org/}{\color{linky}{}}

  \vspace{2mm}

  {\bfseries Software}
  \begin{itemize}
    \setlength\itemsep{0em}
    \item \href{}{\color{linky}{Review}} \ExternalLink
    \item \href{}{\color{linky}{Repository}} \ExternalLink
    \item \href{}{\color{linky}{Archive}} \ExternalLink
  \end{itemize}

  \vspace{2mm}

  {\bfseries Submitted:} \\
  {\bfseries Published:} 

  \vspace{2mm}
  {\bfseries License}\\
  Authors of papers retain copyright and release the work under a Creative Commons Attribution 4.0 International License (\href{http://creativecommons.org/licenses/by/4.0/}{\color{linky}{CC-BY}}).
}

\hypertarget{summary}{%
\section{Summary}\label{summary}}

As a field, (human) behavior genetics explores the sources of individual
differences in psychological traits and characteristics (Galton, 1876)
-- to understand how much of human behavior is from genetic and
environmental factors. More technically, the field focuses on
decomposing the sources of phenotypic variation into genetic (Additive
(A)+ Dominance (D)) and environmental (Shared Environment (C) +
Non-Shared Environment (E)) variance components, by leveraging twin and
family studies . However, these models can do so much more than merely
describe sources of variance; they can be used to infer causation (Burt,
Plaisance, \& Hambrick, 2019).

Although considered the ``gold standard'' for inferring causation
(Rubin, 2008), conducting randomized experiments to tease apart
individual traits from differences in genes and environment are not
always possible. For both ethical and practical considerations,
psychologists often use quasi-experimental designs, which control for
potential known confounds using a variety of statistical approaches
(Garrison \& Rodgers, 2021).

Here, we present software facilitates a genetically-informed
quasi-experimental design, using kinship modeling. It facilitates
discordant-kinship regressions by comparing kin, such as siblings, in a
manner that accounts for gene-and-environmental confounds when examining
causal links in the realm of `nature vs.~nurture.'

\hypertarget{statement-of-need}{%
\section{Statement of Need}\label{statement-of-need}}

Kin-comparison designs distinguish ``within-family variance'' from
``between-family variance'' (Chamberlain \& Griliches, 1975). The former
is a measure of how family members differ from one another; the latter
reflects sources that make family members similar to one another but
distinct from other families (\textbf{garrison2016?}). By partitioning
these sources of variance, behavioral geneticists may greatly reduce
confounds when testing causal hypotheses (Lahey \& D'Onofrio, 2010). Our
R package, \{discord\}, has highly customizable, efficient code for
generating genetically-informed simulations and provides user-friendly
functions to help researchers use sibling-based quasi-experimental
designs.

\{discord\} integrates seamlessly with the NlsyLinks R package, which
provides kinship links for the National Longitudinal Survey of Youth --
a cross-generational, nationally representative survey of over 30,000
participants for up to 35 years (Beasley et al., 2016). It has been used
in multiple studies (cite, Mason, cite!).

\hypertarget{mathematics}{%
\section{Mathematics}\label{mathematics}}

To facilitate kinship comparisons, \{discord\} implements a modified
reciprocal standard dyad model (Kenny, Kashy, \& Cook, 2006) known as
the discordant-kinship model (see (\textbf{garrison2016?}) for a.
Consider the simplified case where a behavioral outcome, \(Y\), is
predicted by one variable, \(X\). The discordant-kinship model relates
the difference in the outcome, \(Y_{i\Delta}\), for the \(i\text{th}\)
kinship pair, where \(\bar{Y}_i\) is the mean level of the outcome,
\(\bar{X}_i\) is the mean level of the predictor, and \(X_{i\Delta}\) is
the between-kin difference in the predictor.

\[
Y_{i\Delta} = \beta_0 + \beta_1 \bar{Y}_i + \beta_2 \bar{X}_i + \beta_3 X_{i\Delta} + \epsilon_i
\]

This model partitions variance in line with the above discussion to
support causal inference. Specifically, the within-family variance is
described by \(Y_{\Delta}\) and \(X_{\Delta}\); between-family variance
is captured by \(\bar{Y}\) and \(\bar{X}\) (Garrison \& Rodgers, 2021).

A non-significant association between \(Y_\Delta\) and \(X_\Delta\)
suggests that the variables are not causally related and may have arisen
from genetic covariance or shared-environmental factors. In contrast, a
significant association may provide support for a causal relationship
between variables depending on the relatedness of each kin pair. That
is, the discordant-kinship model is applicable for any set of kin:
monozygotic twins who share 100\% of their DNA; full-siblings who share
50\%; half-siblings who share 25\%; cousins who share 12.5\%; etc. Thus,
a significant relationship found with monozygotic twins would provide
stronger support for a causal claim than the same relationship between
cousins.

Following (Garrison \& Rodgers, 2021), we recommend interpreting
significant associations as \emph{not disproving a causal relationship}.
Although this design controls for much (sibling) if not all (monozygotic
twins) background heterogeneity, it is possible that a significant
relationship between a phenotype and plausible covariates is possible
due to non-shared environmental influences.

The next section illustrates two examples of discordant-kinship
regressions with the \{discord\} package.

\hypertarget{vaccine-willingness-and-socioeconomic-status}{%
\section{Vaccine willingness and socioeconomic
status}\label{vaccine-willingness-and-socioeconomic-status}}

\hypertarget{introduction}{%
\subsection{Introduction}\label{introduction}}

The following analysis is a pared-down version of previous work
presented at the Behavior Genetics Association 50th Annual Meeting
(Trattner, Kennon, \& Garrison, 2020). The original project was inspired
by reports detailing health disparities amongst ethnic minorities during
the COVID-19 pandemic (Hooper, Nápoles, \& Pérez-Stable, 2020). These
were often attributed to differences in socioeconomic status (SES),
pre-existing health conditions, and COVID-19 symptom severity
(Ssentongo, Ssentongo, Heilbrunn, Ba, \& Chinchilli, 2020; Yang, Gui, \&
Xiong, 2020). In line with the field of behavior genetics, any
intervention to address these disparities must explicitly account for
known gene-and-environmental confounds (Garrison \& Rodgers, 2019;
Williams et al., 2020).

In the original work, we aimed to identify the relationship between SES
and vaccination willingness using a quasi-experimental design. Data came
from the 1979 National Longitudinal Survey of Youth (NLSY79), a
nationally representative household probability sample jointly sponsored
by the U.S. Bureau of Labor Statistics and Department of Defense.
Participants were surveyed annually from 1979 until 1994 at which point
surveys occurred biennially. The data are publicly available at
\url{https://www.nlsinfo.org/} and include responses from a biennial flu
vaccine survey administered between 2006 and 2016. Our work originally
examined whether SES at age 40 is a significant predictor for
vaccination rates using the discordant-kinship model.

As described in (Garrison \& Rodgers, 2019), SES was quantified using
methodology from (Myrianthopoulos \& French, 1968). Individuals were
given a mean quantile score based on their net family income, years of
education, and occupational prestige. Subjects with at least two of the
three quantiles Missing data was imputed from nonmissing components, and
higher scores correspond to higher SES.

The data for this analysis was downloaded with the
\href{https://www.nlsinfo.org/investigator/pages/login}{NLS
Investigator} and can be found
\href{https://github.com/jdtrat/senior-thesis-discord/blob/main/data/flu_shot.dat}{here}.
The SES at age 40 data can be found
\href{https://github.com/jdtrat/senior-thesis-discord/blob/main/data/nlsy-ses.csv}{here}.
For clarity, and to emphasize the functionality of \{discord\}, the data
has been pre-processed using
\href{https://github.com/jdtrat/senior-thesis-discord/blob/main/R/preprocess-discord-flu.R}{this
script}. This discordant-kinship analysis is possible thanks to recent
work that estimated relatedness for approximately 95\% of the NLSY79 kin
pairs (Rodgers et al., 2016). These kinship links are included in the
\href{http://nlsy-links.github.io/NlsyLinks/index.html}{\{NlsyLinks\}} R
package (Beasley et al., 2016) and are easily utilized with the
\{discord\} package.

\hypertarget{data-cleaning}{%
\subsection{Data Cleaning}\label{data-cleaning}}

For this example, we will load the following packages.

\begin{Shaded}
\begin{Highlighting}[]
\CommentTok{\# For easy data manipulation}
\FunctionTok{library}\NormalTok{(dplyr)}
\CommentTok{\# For kinship linkages}
\FunctionTok{library}\NormalTok{(NlsyLinks)}
\CommentTok{\# For discordant{-}kinship regression}
\FunctionTok{library}\NormalTok{(discord)}
\CommentTok{\# To clean dataframe names}
\FunctionTok{library}\NormalTok{(janitor)}
\CommentTok{\# tidyup output}
\FunctionTok{library}\NormalTok{(broom)}

\CommentTok{\# pipe}
\FunctionTok{library}\NormalTok{(magrittr)}
\end{Highlighting}
\end{Shaded}

After some pre-processing, we have a data frame containing subject
identifiers, demographic information such as race and sex, and
behavioral measurements like flu vaccination rates and SES at age 40. A
random slice of this data looks like:

\begin{table}[!h]
\centering
\begin{tabular}[t]{ccccc}
\toprule
CASEID & RACE & SEX & FLU\_total & S00\_H40\\
\midrule
\cellcolor{gray!6}{473} & \cellcolor{gray!6}{0} & \cellcolor{gray!6}{1} & \cellcolor{gray!6}{0} & \cellcolor{gray!6}{23.38833}\\
296 & 0 & 1 & 0 & 63.23976\\
\cellcolor{gray!6}{387} & \cellcolor{gray!6}{1} & \cellcolor{gray!6}{1} & \cellcolor{gray!6}{3} & \cellcolor{gray!6}{64.19161}\\
83 & 1 & 1 & 0 & 46.41507\\
\cellcolor{gray!6}{529} & \cellcolor{gray!6}{0} & \cellcolor{gray!6}{0} & \cellcolor{gray!6}{0} & \cellcolor{gray!6}{33.99638}\\
\addlinespace
29 & 0 & 0 & 2 & 56.80481\\
\bottomrule
\end{tabular}
\end{table}

Using the kinship relationships from the \{NlsyLinks\} package, we can
create a data frame that lends itself to discordant analysis. For each
kin pair, the function \texttt{CreatePairLinksSingleEntered()} takes a
data set like the one above, \textbf{{[}a specification of the NLSY
database and the kin's relatedness{]}}, and the variables of interest.
It returns a data frame where every row is a kin-pair and each column is
a variable of interest with a suffix indicating to which individual the
value corresponds.

For this example, we will examine the relationship between flu
vaccinations received between 2006-2016 and SES at age 40 between full
siblings. As such, we specify the following variables from the
pre-processed data frame previewed above.

\begin{Shaded}
\begin{Highlighting}[]
\CommentTok{\# Get kinship links for individuals with the following variables:}
\NormalTok{link\_vars }\OtherTok{\textless{}{-}} \FunctionTok{c}\NormalTok{(}\StringTok{"FLU\_total"}\NormalTok{, }\StringTok{"FLU\_2008"}\NormalTok{, }\StringTok{"FLU\_2010"}\NormalTok{, }
               \StringTok{"FLU\_2012"}\NormalTok{, }\StringTok{"FLU\_2014"}\NormalTok{, }\StringTok{"FLU\_2016"}\NormalTok{, }
               \StringTok{"S00\_H40"}\NormalTok{, }\StringTok{"RACE"}\NormalTok{, }\StringTok{"SEX"}\NormalTok{)}
\end{Highlighting}
\end{Shaded}

We now link the subjects by the specified variables using
\texttt{CreatePairLinksSingleEntered()}, from the NlsyLinks

\begin{Shaded}
\begin{Highlighting}[]
\CommentTok{\# Specify NLSY database and kin relatedness }
\NormalTok{link\_pairs }\OtherTok{\textless{}{-}}\NormalTok{ Links79PairExpanded }\SpecialCharTok{\%\textgreater{}\%}
  \FunctionTok{filter}\NormalTok{(RelationshipPath }\SpecialCharTok{==} \StringTok{"Gen1Housemates"} \SpecialCharTok{\&}\NormalTok{ RFull }\SpecialCharTok{==} \FloatTok{0.5}\NormalTok{)}

\NormalTok{df\_link }\OtherTok{\textless{}{-}} \FunctionTok{CreatePairLinksSingleEntered}\NormalTok{(}\AttributeTok{outcomeDataset =}\NormalTok{ flu\_ses\_data,}
                                        \AttributeTok{linksPairDataset =}\NormalTok{ link\_pairs,}
                                        \AttributeTok{outcomeNames =}\NormalTok{ link\_vars)}
\end{Highlighting}
\end{Shaded}

We have saved this data frame as \texttt{df\_link}. A random subset of
this data is:

\begin{table}[!h]
\centering
\resizebox{\linewidth}{!}{
\begin{tabular}[t]{ccccccc}
\toprule
ExtendedID & SubjectTag\_S1 & SubjectTag\_S2 & FLU\_total\_S1 & FLU\_total\_S2 & S00\_H40\_S1 & S00\_H40\_S2\\
\midrule
\cellcolor{gray!6}{1514} & \cellcolor{gray!6}{151400} & \cellcolor{gray!6}{151500} & \cellcolor{gray!6}{0} & \cellcolor{gray!6}{1} & \cellcolor{gray!6}{62.32863} & \cellcolor{gray!6}{45.49521}\\
1438 & 143800 & 143900 & 2 & 1 & 25.88258 & 36.66835\\
\cellcolor{gray!6}{1252} & \cellcolor{gray!6}{125200} & \cellcolor{gray!6}{125300} & \cellcolor{gray!6}{0} & \cellcolor{gray!6}{1} & \cellcolor{gray!6}{80.68968} & \cellcolor{gray!6}{49.19417}\\
658 & 65800 & 65900 & 2 & 3 & 48.38794 & 64.38859\\
\cellcolor{gray!6}{1189} & \cellcolor{gray!6}{119100} & \cellcolor{gray!6}{119200} & \cellcolor{gray!6}{0} & \cellcolor{gray!6}{2} & \cellcolor{gray!6}{87.65669} & \cellcolor{gray!6}{45.92817}\\
\addlinespace
1732 & 173200 & 173300 & 0 & 5 & 66.92792 & 81.42641\\
\bottomrule
\end{tabular}}
\end{table}

Notice that, with the exception of the first column indicating the
specific pair, each column name has the suffix ``\_S1'' and ``\_S2.'' As
mentioned above, these suffices identify which sibling the column values
correspond.

This data is almost ready for analysis, but we want to ensure that the
data are representative of actual trends. The \texttt{FLU\_total} column
is simply a sum of the biennial survey responses. So for a given
sibling-pair, one or both individuals may not have responded to the
survey indicating their vaccination status. If that's the case, we want
to exclude those siblings to reduce \textbf{{[}non-response bias{]}}. We
can do this by examining the biennial responses and removing any rows
that have \texttt{NA}.

\begin{Shaded}
\begin{Highlighting}[]
\CommentTok{\# Take the linked data, group by the sibling pairs and}
\CommentTok{\# count the number of responses for flu each year. If there is an NA, }
\CommentTok{\# then data is missing for one of the years, and we omit it.}
\NormalTok{consistent\_kin }\OtherTok{\textless{}{-}}\NormalTok{ df\_link }\SpecialCharTok{\%\textgreater{}\%} 
  \FunctionTok{group\_by}\NormalTok{(SubjectTag\_S1, SubjectTag\_S2) }\SpecialCharTok{\%\textgreater{}\%} 
  \FunctionTok{count}\NormalTok{(FLU\_2008\_S1, FLU\_2010\_S1, }
\NormalTok{        FLU\_2012\_S1, FLU\_2014\_S1, }
\NormalTok{        FLU\_2016\_S1, FLU\_2008\_S2, }
\NormalTok{        FLU\_2010\_S2, FLU\_2012\_S2, }
\NormalTok{        FLU\_2014\_S2, FLU\_2016\_S2) }\SpecialCharTok{\%\textgreater{}\%} 
  \FunctionTok{na.omit}\NormalTok{()}

\CommentTok{\# Create the flu\_modeling\_data object with only consistent responders.}
\CommentTok{\# Clean the column names with the \{janitor\} package.}
\NormalTok{flu\_modeling\_data }\OtherTok{\textless{}{-}} \FunctionTok{semi\_join}\NormalTok{(df\_link, }
\NormalTok{                               consistent\_kin, }
                               \AttributeTok{by =} \FunctionTok{c}\NormalTok{(}\StringTok{"SubjectTag\_S1"}\NormalTok{, }
                                      \StringTok{"SubjectTag\_S2"}\NormalTok{)) }\SpecialCharTok{\%\textgreater{}\%}
  \FunctionTok{clean\_names}\NormalTok{()}
\end{Highlighting}
\end{Shaded}

To avoid violating assumptions of independence, in our analysis we
specify that the sibling-pairs should be from unique households (i.e.~we
randomly select one sibling pair per household).

\begin{Shaded}
\begin{Highlighting}[]
\NormalTok{flu\_modeling\_data }\OtherTok{\textless{}{-}}\NormalTok{ flu\_modeling\_data }\SpecialCharTok{\%\textgreater{}\%}
  \FunctionTok{group\_by}\NormalTok{(extended\_id) }\SpecialCharTok{\%\textgreater{}\%}
  \FunctionTok{slice\_sample}\NormalTok{() }\SpecialCharTok{\%\textgreater{}\%}
  \FunctionTok{ungroup}\NormalTok{()}
\end{Highlighting}
\end{Shaded}

The data we will use for modeling now contains additional information
for each kin pair, including sex and race of each individual, flu
vaccination status for the biennial survey between 2006-2016, and a
total flu vaccination count for that period. The total vaccination count
ranges from 0 - 5, where 0 indicates that the individual did not get a
vaccine in any year between 2006-2016 and 5 indicates that an individual
got at least 5 vaccines between 2006-2016. Although our data set has
individual years, we are only interested in the total. A subset of the
data to use in this regression looks like:

\begin{table}[!h]
\centering
\resizebox{\linewidth}{!}{
\begin{tabular}[t]{ccccccccccc}
\toprule
extended\_id & subject\_tag\_s1 & subject\_tag\_s2 & flu\_total\_s1 & flu\_total\_s2 & race\_s1 & race\_s2 & sex\_s1 & sex\_s2 & s00\_h40\_s1 & s00\_h40\_s2\\
\midrule
\cellcolor{gray!6}{17} & \cellcolor{gray!6}{1700} & \cellcolor{gray!6}{1800} & \cellcolor{gray!6}{0} & \cellcolor{gray!6}{0} & \cellcolor{gray!6}{0} & \cellcolor{gray!6}{0} & \cellcolor{gray!6}{1} & \cellcolor{gray!6}{1} & \cellcolor{gray!6}{49.26537} & \cellcolor{gray!6}{74.92440}\\
29 & 2900 & 3000 & 2 & 0 & 0 & 0 & 0 & 0 & 56.80481 & 32.05423\\
\cellcolor{gray!6}{37} & \cellcolor{gray!6}{3700} & \cellcolor{gray!6}{3800} & \cellcolor{gray!6}{1} & \cellcolor{gray!6}{5} & \cellcolor{gray!6}{0} & \cellcolor{gray!6}{0} & \cellcolor{gray!6}{0} & \cellcolor{gray!6}{0} & \cellcolor{gray!6}{58.55547} & \cellcolor{gray!6}{50.45408}\\
40 & 4000 & 4100 & 2 & 0 & 0 & 0 & 1 & 1 & 78.19220 & 73.41860\\
\cellcolor{gray!6}{58} & \cellcolor{gray!6}{5800} & \cellcolor{gray!6}{5900} & \cellcolor{gray!6}{5} & \cellcolor{gray!6}{0} & \cellcolor{gray!6}{0} & \cellcolor{gray!6}{0} & \cellcolor{gray!6}{0} & \cellcolor{gray!6}{1} & \cellcolor{gray!6}{80.56835} & \cellcolor{gray!6}{49.68414}\\
\addlinespace
61 & 6100 & 6200 & 3 & 4 & 0 & 0 & 0 & 0 & 74.43720 & 50.56920\\
\cellcolor{gray!6}{67} & \cellcolor{gray!6}{6700} & \cellcolor{gray!6}{6800} & \cellcolor{gray!6}{4} & \cellcolor{gray!6}{4} & \cellcolor{gray!6}{0} & \cellcolor{gray!6}{0} & \cellcolor{gray!6}{1} & \cellcolor{gray!6}{0} & \cellcolor{gray!6}{89.67767} & \cellcolor{gray!6}{82.68649}\\
74 & 7500 & 7600 & 0 & 0 & 0 & 0 & 0 & 1 & 88.15524 & 61.54234\\
\cellcolor{gray!6}{83} & \cellcolor{gray!6}{8300} & \cellcolor{gray!6}{8400} & \cellcolor{gray!6}{0} & \cellcolor{gray!6}{3} & \cellcolor{gray!6}{1} & \cellcolor{gray!6}{1} & \cellcolor{gray!6}{1} & \cellcolor{gray!6}{1} & \cellcolor{gray!6}{46.41507} & \cellcolor{gray!6}{64.12765}\\
85 & 8600 & 8700 & 0 & 4 & 1 & 1 & 0 & 1 & 45.06552 & 64.14045\\
\bottomrule
\end{tabular}}
\end{table}

\hypertarget{modeling-and-interpretation}{%
\subsection{Modeling and
Interpretation}\label{modeling-and-interpretation}}

To perform the regression using the \{discord\} package, we supply the
data frame and specify the outcome and predictors. It also requires a
kinship pair id, \texttt{extended\_id} in our case, as well as pair
identifiers -- the column name suffixes that identify to which kin a
column's values correspond (``\_s1'' and ``\_s2'' in our
case).\footnote{Note these were previously ``\_S1'' and ``\_S2,''
  however, we used the \texttt{clean\_names()} function which coerced
  the column names to lowercase.} Optional, though recommended, are
columns containing sex and race information to control for as additional
covariates. In our case, these columns are prefixed ``race'' and
``sex.'' Per the
\href{https://github.com/jdtrat/senior-thesis-discord/blob/main/R/preprocess-discord-flu.R}{pre-processing
script}, these columns contain dummy variables where the reference group
for race is non-Black, non-Hispanic and the reference group for sex is
female.

By entering this information into the \texttt{discord\_regression()}
function, we can run the model as such:

\begin{Shaded}
\begin{Highlighting}[]
\CommentTok{\# Setting a seed for reproducibility}
\FunctionTok{set.seed}\NormalTok{(}\DecValTok{18}\NormalTok{)}
\NormalTok{flu\_model\_output }\OtherTok{\textless{}{-}} \FunctionTok{discord\_regression}\NormalTok{(}
                          \AttributeTok{data =}\NormalTok{ flu\_modeling\_data,}
                          \AttributeTok{outcome =} \StringTok{"flu\_total"}\NormalTok{,}
                          \AttributeTok{predictors =} \StringTok{"s00\_h40"}\NormalTok{,}
                          \AttributeTok{id =} \StringTok{"extended\_id"}\NormalTok{,}
                          \AttributeTok{sex =} \StringTok{"sex"}\NormalTok{,}
                          \AttributeTok{race =} \StringTok{"race"}\NormalTok{,}
                          \AttributeTok{pair\_identifiers =} \FunctionTok{c}\NormalTok{(}\StringTok{"\_s1"}\NormalTok{, }\StringTok{"\_s2"}\NormalTok{)}
\NormalTok{                          )}
\end{Highlighting}
\end{Shaded}

The default output of \texttt{discord\_regression()} is an \texttt{lm}
object. The metrics for our regression can be summarized as follows:

\begin{table}[!h]
\centering
\begin{tabular}[t]{ccccc}
\toprule
Term & Estimate & Standard Error & T Statistic & P Value\\
\midrule
\cellcolor{gray!6}{(Intercept)} & \cellcolor{gray!6}{1.326} & \cellcolor{gray!6}{0.191} & \cellcolor{gray!6}{6.940} & \cellcolor{gray!6}{p<0.001}\\
flu\_total\_mean & 0.204 & 0.034 & 6.076 & p<0.001\\
\cellcolor{gray!6}{s00\_h40\_diff} & \cellcolor{gray!6}{0.008} & \cellcolor{gray!6}{0.002} & \cellcolor{gray!6}{3.927} & \cellcolor{gray!6}{p<0.001}\\
s00\_h40\_mean & 0.004 & 0.003 & 1.376 & p=0.169\\
\cellcolor{gray!6}{sex\_1} & \cellcolor{gray!6}{-0.135} & \cellcolor{gray!6}{0.097} & \cellcolor{gray!6}{-1.388} & \cellcolor{gray!6}{p=0.165}\\
\addlinespace
race\_1 & -0.053 & 0.102 & -0.516 & p=0.606\\
\cellcolor{gray!6}{sex\_2} & \cellcolor{gray!6}{0.057} & \cellcolor{gray!6}{0.097} & \cellcolor{gray!6}{0.590} & \cellcolor{gray!6}{p=0.555}\\
\bottomrule
\end{tabular}
\end{table}

Looking at this output, the intercept can be thought of as the average
difference in outcomes between siblings, ignoring all other variables.
That is, it looks like the average difference for two sisters of a
non-minority ethnic background (the default sex and race values) is
approximately 1.3. The term \texttt{flu\_total\_mean} is essentially an
extra component of the intercept that captures some non-linear trends
and allows the difference score to change as a function of the average
predictors. Here, this is the mean socioeconomic status for the
siblings, \texttt{s00\_h40\_mean}. we also accounted for sex and race,
neither of which have a statistically significant effect on the
differences in flu vaccine shots between siblings (different families)
or within a sibling pair (same family).

The most important metric from the output, though, is the difference
score, \texttt{s00\_h40\_diff}. Here, it is statistically significant.
An interpretation of this might be, ``the difference in socioeconomic
status between siblings at age 40 is positively associated with the
difference in the number of flu vaccinations received between
2006-2016.'' This means that a sibling with 10\% higher SES is expected
to have 0.07846 more flu shots.

The goal of performing a discordant-kinship regression is to see whether
there is a significant difference in some behavioral measure while
controlling for as much gene-and-environmental variance as possible. In
this section, we walked-through an analysis showing a statistically
significant difference in the number of flu shots a sibling received and
their socioeconomic status. From this, we \emph{could not} claim the
relationship is causal. However, we cannot eliminate causality because
there are statistically significant within- and between-family
differences in our predictors and outcomes.

\hypertarget{conclusion}{%
\section{Conclusion}\label{conclusion}}

In its current implementation, the \{discord\} package encourages best
practices for performing discordant-kinship regressions. For example,
the main function has the default expectation that sex and race
indicators will be supplied. These measures are both important
covariates when testing for causality between familial background and
psychological characteristics.

This, and other design choices, are crucial to facilitating transparent
and reproducible results. Software ever-evolves, however, and to further
support reproducible research we plan to provide improved documentation
and allow for easier inspection of the underlying model implementation
and results.

\hypertarget{acknowledgements}{%
\section{Acknowledgements}\label{acknowledgements}}

We acknowledge contributions from Cermet Ream, Joe Rodgers, and support
from Lucy D'Agostino McGowan on this project.

\hypertarget{references}{%
\section*{References}\label{references}}
\addcontentsline{toc}{section}{References}

\hypertarget{refs}{}
\begin{CSLReferences}{1}{0}
\leavevmode\hypertarget{ref-beasley2016}{}%
Beasley, W., Rodgers, J., Bard, D., Hunter, M., Garrison, S. M., \&
Meredith, K. (2016). \emph{NlsyLinks: Utilities and kinship information
for research with the NLSY}. Retrieved from
\url{https://CRAN.R-project.org/package=NlsyLinks}

\leavevmode\hypertarget{ref-burt2019}{}%
Burt, S. A., Plaisance, K. S., \& Hambrick, D. Z. (2019). Understanding
{``}What Could Be{''}: A Call for {`}Experimental Behavioral
Genetics{'}. \emph{Behavior Genetics}, \emph{49}(2), 235--243.
doi:\href{https://doi.org/10.1007/s10519-018-9918-y}{10.1007/s10519-018-9918-y}

\leavevmode\hypertarget{ref-chamberlain1975}{}%
Chamberlain, G., \& Griliches, Z. (1975). Unobservables with a
Variance-Components Structure: Ability, Schooling, and the Economic
Success of Brothers. \emph{International Economic Review}, \emph{16}(2),
422--449.

\leavevmode\hypertarget{ref-garrison2019}{}%
Garrison, S. M., \& Rodgers, J. L. (2019). Decomposing the causes of the
socioeconomic status-health gradient with biometrical modeling.
\emph{Journal of Personality and Social Psychology}, \emph{116}(6),
1030--1047.
doi:\href{https://doi.org/10.1037/pspp0000226}{10.1037/pspp0000226}

\leavevmode\hypertarget{ref-garrison2021}{}%
Garrison, S. M., \& Rodgers, J. L. (2021, June 28). \emph{Using
genetically-informed designs to test causal claims without experiments:
Discordant-sibling designs and applications for differential
psychology.}

\leavevmode\hypertarget{ref-hooper2020}{}%
Hooper, M. W., Nápoles, A. M., \& Pérez-Stable, E. J. (2020). COVID-19
and Racial/Ethnic Disparities. \emph{JAMA}.
doi:\href{https://doi.org/10.1001/jama.2020.8598}{10.1001/jama.2020.8598}

\leavevmode\hypertarget{ref-kenny2006}{}%
Kenny, D. A., Kashy, D. A., \& Cook, W. L. (2006). \emph{Dyadic data
analysis}. Dyadic data analysis. New York, NY, US: Guilford Press.

\leavevmode\hypertarget{ref-lahey2010}{}%
Lahey, B. B., \& D'Onofrio, B. M. (2010). All in the Family: Comparing
Siblings to Test Causal Hypotheses Regarding Environmental Influences on
Behavior. \emph{Current Directions in Psychological Science},
\emph{19}(5), 319--323.
doi:\href{https://doi.org/10.1177/0963721410383977}{10.1177/0963721410383977}

\leavevmode\hypertarget{ref-myrianthopoulos1968}{}%
Myrianthopoulos, N. C., \& French, K. S. (1968). An application of the
U.S. Bureau of the Census socioeconomic index to a large, diversified
patient population. \emph{Social Science and Medicine}, \emph{2}(3),
283--299.
doi:\href{https://doi.org/10.1016/0037-7856(68)90004-8}{10.1016/0037-7856(68)90004-8}

\leavevmode\hypertarget{ref-rodgers2016}{}%
Rodgers, J. L., Beasley, W. H., Bard, D. E., Meredith, K. M., Hunter, M.
D., Johnson, A. B., Buster, M., et al. (2016). The NLSY kinship links:
Using the NLSY79 and NLSY-children data to conduct genetically-informed
and family-oriented research. \emph{Behavior genetics}, \emph{46}(4),
538--551.
doi:\href{https://doi.org/10.1007/s10519-016-9785-3}{10.1007/s10519-016-9785-3}

\leavevmode\hypertarget{ref-rubin2008}{}%
Rubin, D. B. (2008). For objective causal inference, design trumps
analysis. \emph{The Annals of Applied Statistics}, \emph{2}(3),
808--840.
doi:\href{https://doi.org/10.1214/08-AOAS187}{10.1214/08-AOAS187}

\leavevmode\hypertarget{ref-ssentongo2020}{}%
Ssentongo, P., Ssentongo, A. E., Heilbrunn, E. S., Ba, D. M., \&
Chinchilli, V. M. (2020). The association of cardiovascular disease and
other pre-existing comorbidities with COVID-19 mortality: A systematic
review and meta-analysis. \emph{medRxiv}, 2020.05.10.20097253.
doi:\href{https://doi.org/10.1101/2020.05.10.20097253}{10.1101/2020.05.10.20097253}

\leavevmode\hypertarget{ref-jonathantrattner2020}{}%
Trattner, J., Kennon, L., \& Garrison, S. M. (2020). Vaccine willingness
and socioeconomic status: a biometrically controlled design.
\emph{Behavior Genetics}, Behavior Genetics Association 50th Annual
Meeting Abstracts, \emph{50}(6), 483--483.
doi:\href{https://doi.org/10.1007/s10519-020-10018-8}{10.1007/s10519-020-10018-8}

\leavevmode\hypertarget{ref-williams2020}{}%
Williams, F. M., Freydin, M., Mangino, M., Couvreur, S., Visconti, A.,
Bowyer, R. C., Roy, C. I. L., et al. (2020). Self-reported symptoms of
covid-19 including symptoms most predictive of SARS-CoV-2 infection, are
heritable. \emph{medRxiv}, 2020.04.22.20072124.
doi:\href{https://doi.org/10.1101/2020.04.22.20072124}{10.1101/2020.04.22.20072124}

\leavevmode\hypertarget{ref-yang2020}{}%
Yang, R., Gui, X., \& Xiong, Y. (2020). Comparison of Clinical
Characteristics of Patients with Asymptomatic vs Symptomatic Coronavirus
Disease 2019 in Wuhan, China. \emph{JAMA Network Open}, \emph{3}(5),
e2010182--e2010182.
doi:\href{https://doi.org/10.1001/jamanetworkopen.2020.10182}{10.1001/jamanetworkopen.2020.10182}

\end{CSLReferences}

\end{document}
